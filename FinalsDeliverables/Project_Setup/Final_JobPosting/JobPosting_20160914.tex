%%%%%%%%%%%%%%%%%%%%%%%%%%%%%%%%%%%%%%%%%
% Title: 				Job Posting for LANDIS Updates/Bug Fixes
% Project Descriptor:	LANDIS Programming Updates Project
% Project ID:			2016SoE021
% Author:				bmarron
% Origin Date:			14 Sep 2016
% Revision Date:		19 Sep 2016
%
%
%%%%%%%%%%%%%%%%%%%%%%%%%%%%%%%%%%%%%%%%%
%----------------------------------------------------------------------------------------
%	PACKAGES AND OTHER DOCUMENT CONFIGURATIONS
%----------------------------------------------------------------------------------------

\documentclass[letterpaper, 12pt]{article}

%---------------------Required packages
\usepackage{lipsum}
\usepackage{fancyhdr} % Required for custom headers
\usepackage{lastpage} % Required to determine the last page for the footer
\usepackage{extramarks} % Required for headers and footers
\usepackage{graphicx} % Required to insert images
\usepackage{hyperref}
\usepackage{amsmath}
\usepackage{amsthm}
\usepackage{amssymb}
\usepackage{apacite}
\usepackage[english]{babel}
\usepackage{comment}
\usepackage{multirow}
\usepackage[all]{nowidow}
\usepackage{longtable}
\usepackage{etoolbox}
\setlength{\LTcapwidth}{=6.95in} %longtable caption width goes the full textwidth 
\usepackage[sc]{mathpazo} % use Palatino for the text and the Pazo fonts for math
\linespread{1.05} % Line spacing - Palatino needs more space between lines
\usepackage[T1]{fontenc} % Use 8-bit encoding that has 256 glyphs
	%\usepackage[utf8]{inputenc} % alternate fonts
\usepackage{microtype} % Slightly tweak font spacing for aesthetics
\usepackage[left=1 in, right=1 in, top=.5 in]{geometry} 

%--------------------Special environments
\newenvironment{myindentpar}[1]%  paragraph indent, if needed
   {\begin{list}{}%
       {\setlength{\leftmargin}{#1}}%
           \item[]%
   }
     {\end{list}}

%----------------Optional headers/footers
%\pagestyle{fancy}
%\lhead{} % Top left header
%\chead{} % Top center header
%\rhead{bmarron} % Top right header
%\lfoot{\lastxmark} % Bottom left footer
%\cfoot{} % Bottom center footer
%\rfoot{Page\ \thepage\ of\ \pageref{LastPage}} % Bottom right footer
%\renewcommand\headrulewidth{0.4pt} % Size of the header rule
%\renewcommand\footrulewidth{0.4pt} % Size of the footer rule



\setlength\parindent{0pt} % Removes all indentation from paragraphs
\setcounter{secnumdepth}{0} % Removes default section numbers

   

%----------------------------------------
%	BEGIN DOC
%----------------------------------------
\begin{document}
\section{Job Posting}
\vspace*{1cm}
\textbf{Mid-level C\# developer - \$20/hr, part-time, working as part of a team to provide updates and corrections to the LANDIS-II forest landscape change model.} \\
\\
The Dynamic Ecosystems and Landscapes Lab, under the direction of Dr. Robert Scheller, has immediate need for three (3) mid-level C\# developers to implement updates and corrections to the LANDIS-II forest landscape change model and its various extensions. LANDIS-II is an open-source, spatially-explicit, process-based simulation model of LANdscape DIsturbance and succession. LANDIS-II is sponsored by the LANDIS-II Foundation, a not-for-profit 501(3)(c) scientific and educational organization (http://www.landis-ii.org/home). \\

\textbf{Summary:}\\

Developers (C\#) are responsible for modification and generation of well-commented code that accurately implements updates and corrections to the LANDIS-II core model and its various extensions as defined by open tickets from key scientific users.  Developers are expected to use sound engineering practices and are expected to perform and participate in all necessary QA/QC procedures and practices including code review, code testing, task branching, and work logs to produce well-documented, model-verified results. Developers must be comfortable working autonomously as well as within a collaborative problem solving environment. Developers are expected to work 10-15 hr/wk throughout the anticipated 30 weeks of the project. Expected start date is October 10, 2016.\\

\textbf{Responsibilities:}
\begin{myindentpar}{1em}
* Access the LANDIS-II Foundation's GitHub repository to self-select job tasks from open ticket items (Issues on GitHub)\\
* Analyze open ticket change requests in light of C\# and the LANDIS-II modeling environment\\
* Work autonomously to design draft coding solutions to open ticket items\\
* Work collectively with other Developers and the Project Manager to perform code review and testing to produce final coding solutions to open ticket items\\
* Rigorously and consistently implement all project QA/QC procedures and practices including revision control, work logs, unit testing and code reviews, by and for peers
\end{myindentpar}

\newpage

\textbf{Skills:}
\begin{myindentpar}{1em}
* Strong understanding of object-oriented programming \\
* Strong understanding of iterative problem solving (analysis, design, coding, testing, repeat) \\
* Knack for writing clean, well-commented, and readable C\# code \\
* Proficient in C\#, with a good knowledge of its ecosystems and architectures \\
* Experience with Git and GitHub as code versioning control tools \\
* Experience with IDEs for C\# (Visual Studio 2015 or MonoDevelop) \\
* Experience with JavaScript a plus \\
\end{myindentpar}
	
Interested? Please send a resume, a small portfolio of applicable work samples or the URL to your publicly accessible repository, and a brief cover letter that includes your contact information to:\\

\textit{Bruce Marron, Project Manager\\
bmarron@pdx.edu} \\

All submitted documents should be in .pdf format.







%--------Optional references
%\renewcommand{\refname}{\normalfont\selectfont\small \textbf{References}} 
%\bibliographystyle{/usr/local/share/texmf/tex/latex/apacite/apacite}
%\bibliography{/home/bmarron//Desktop/BibTex/My_Library_20160725}

\end{document}
